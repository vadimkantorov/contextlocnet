
%%

In both computer vision and machine learning, there has been a large body of recent research on WSL~\cite{Anonymous:2007fd,shi2012transfer,siva2012defence,Deselaers:2012ci,Siva2013,song2014learning,song14slsvm,bilen2014weakly,bilen2015weakly,Wang:2014tg,Cinbis:2015wn,Oquab:2015us,Bilen:2015uo,Jaderberg:2015vo,Zhou:2015wx}. 
Such methods typically attempt
to localize objects in the form of bounding boxes with visually consistent
appearance in the training images, where multiple objects in different viewpoints
and configurations appear in cluttered backgrounds.  Most of existing approaches
to WSL are formulated as or are closely related to multiple instance learning
(MIL)~\cite{long1998pac}, where each positive image has at least one true
bounding box for a target class, and negative images contain false boxes only.
They typically alternate between
estimating a discriminative representation of the object and selecting an object
box in positive images based on this representation. Since the task consists
in a non-convex optimization problem, WSL has focused on robust initialization
and effective regularization strategies. 

Chum and Zisserman~\cite{Anonymous:2007fd} initialize candidate boxes 
using discriminative visual words, and  update localization by maximizing
the average pairwise similarity across the positive images.
Shi~\etal\cite{shi2012transfer} introduce  %%
the Latent Dirichlet Allocation (LDA) topic model for WSL, and 
Siva~\etal\cite{siva2012defence} %%
%%
%%
propose an effective negative mining approach combined
with discriminative saliency measures.  Deselaers~\etal\cite{Deselaers:2012ci} instead
initialize candidate boxes using the objectness method~\cite{Anonymous:2012kg}, 
and %%
propose a CRF-based model that jointly localizes objects in positive training
images. %%
%%
%%
%%
%%
Song~\etal formulate 
an initialization strategy for WSL as a discriminative submodular cover problem in
a graph-based framework~\cite{song2014learning}, and develop a negative mining technique to increase robustness against incorrectly localized boxes~\cite{song14slsvm}.  Bilen~\etal
~\cite{bilen2014weakly} propose a relaxed version of MIL that softly
labels object instances instead of choosing the highest scoring ones.
In~\cite{bilen2015weakly}, they also propose a discriminative convex clustering
algorithm to jointly learn a discriminative object model and enforce the
similarity of the localized object regions.
%%
%%
%%
Wang~\etal\cite{Wang:2014tg} propose an iterative latent semantic
clustering algorithm based on latent Semantic Analysis (pLSA) that 
selects the most discriminative cluster for each class in terms of its
classification performance. 
Cinbis~\etal\cite{Cinbis:2015wn} extend a standard MIL approach and propose a
multi-fold strategy that splits the training data to escape bad local optima. 

As CNNs have turned out to be surprisingly effective in many vision tasks
including classification and detection,  recent state-of-the-art WSL approaches
also build on CNN
architectures~\cite{Oquab:2015us,Bilen:2015uo,Jaderberg:2015vo,Zhou:2015wx} or
CNN features~\cite{Wang:2014tg,Cinbis:2015wn}.
Cinbis~\etal\cite{Cinbis:2015wn} combine multi-fold multiple-instance learning
with CNN features. Wang~\etal\cite{Wang:2014tg} develop a semantic
clustering method on top of pretrained CNN features. While these methods produce
promising results, they are not trained end-to-end.
Oquab~\etal\cite{Oquab:2015us} propose a CNN architecture with global max
pooling on top of its final convolutional layer.  Zhou~\etal\cite{Zhou:2015wx}
apply global average pooling instead to encourage the network to cover the full
extent of the object. Rather than directly providing the full extent of the
object, however, these pooling-based approaches are limited to a position of a
discriminative part or require a separate post-processing step to obtain the final
localization. Jaderberg \etal~\cite{Jaderberg:2015vo} propose a CNN architecture
with spatial transformer layers that automatically transform spatial feature
maps to align objects to a common reference frame. Bilen~\etal
~\cite{Bilen:2015uo} modify a region-based CNN
architecture~\cite{Girshick_2015_ICCV} and propose a CNN with two streams, one
focusing on recognition and the other one on localization, that performs
simultaneously region selection and classification.  Our work is related to
these CNN-based MIL approaches that perform WSL by end-to-end training from
image-level labels.  In contrast to the above methods, however, we focus on a
context-aware CNN architecture that exploits contextual relation between a
candidate region and its surrounding regions. 

While contextual information has been widely employed for object
detection~\cite{Oliva:2007ui,Rabinovich:2007wy,Felzenszwalb:2009wx,Girshick:2016ig,Gidaris:2015cx},
the use of context has received relatively little attention in weakly supervised or
unsupervised localization.  Russakovsky~\etal\cite{Russakovsky:2012wj} and
Cinbis~\etal\cite{Cinbis:2015wn} use a background descriptor computed over
features outside a candidate box, and demonstrate that background modelling can
improve WSL as compared to foreground modelling only. %%
%%
Doersch~\etal\cite{Gupta:xu09Pf6c} align contextual regions of an object patch
to gradually discovers a visual object cluster in their method of iterative
region prediction and context alignment. Cho~\etal\cite{Cho:2015vz,kwak2015unsupervised} propose a
contrast-based contextual score for unsupervised object localization, which measures the contrast of
matching scores between a candidate region and its surrounding candidate
regions. Our context-aware CNN models are inspired by these previous approaches. 
%%
%%
%%
%%
%%
%%
%%
%%
%%
We would like to emphasize that while the use of contextual information is not new in itself, we apply it to build a novel CNN architecture for WSL, that is, to
the best of our knowledge, unique to our work. We believe that the simplicity of
our basic models makes them extendable to a variety of weakly supervised
computer vision tasks for more accurate localization and learning.
%%
